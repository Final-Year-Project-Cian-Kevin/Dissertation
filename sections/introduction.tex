\chapter{Introduction}
During our first three years in GMIT (Galway Mayo Institute of Technology) we were taught a broad range of topics from hardware to software. This was done to get us ready for whatever facet we chose. As part of this project is to show what we learned and put it into practice.

In late 2018, in the first semester of our fourth year, we were told we would be doing a group project over the two semesters. So in October 2018 Kevin Barry and Cian Gannon decided to form a group and started brainstorming ideas in order to start work early. Bringing those ideas to our project supervisors and getting their feedback on our idea we were able to form the idea for our final year project. With input from supervisors and using our own ideas, we were able to move forward with the technologies we would use as our foundation. This was an extremely important decision for us as we wanted to pick relevant technologies and keep the scope within the scope of the level 8 course. 

In our first week of the first year, our student union representatives created a group page on Facebook. This was something which kept everyone in the course in contact with one another outside of college hours. Social media has rapidly expanded in the last ten years, where it's now quite hard to find someone who doesn't have some sort of social media account on one of the numerous platforms. Our world has become ever increasingly dependent on social media and the features it brings.

Looking at social media and how reliant our world has become on it, we wanted to understand it better and have a stronger grasp on the underlying technologies.

\section{Project Objectives} \label{objectives}
As mentioned above, our main goal for this project was to increase our understanding of how social media sites operate and the technologies used to make them.

This group project is divided into two parts, the research-based dissertation and the applied project we will be discussing the project in terms of the research behind it and the technologies used to build it.

\section{Metrics for success or failure}
In order to create the application in a controlled and easy to manage progress in the early stages of this project it was crucial that we outlined the metrics for success or failure. Doing so, enabled us to keep on track with the core focus of what we were attempting to achieve. The metrics we defined related closely to the the objectives outlined in \ref{objectives}. The simplified list of metrics for success from both a dissertation and web application is as follow:
\begin{itemize}
  \item \textit{A concise, yet comprehensive dissertation which can be understood by anyone regardless of the initial knowledge of the technologies implemented.} To measure this, as each section of the dissertation was roughy drafted up, we released sections to friends and fellow students to read. We would then process their feedback and make alterations were required.  
  \item \textit{A Simple, easy to use, functional web application}. To measure this we followed the same steps as above, releasing beta versions to friends of different technological abilities, allowing us to receive feedback both from a technical and non technical perspective.
  \item \textit{A social platform that actually develops an online community}.To measure the effectiveness of the application from a social point of view.We kept a log of the user activities while monitoring the site. Thus enabling us to at certain time points be able to see users following more users and interacting with a larger reach.
  \item \textit{Teamwork Collaboration} We felt from a project management perspective and by researching the importance of a good team in industry, that we would measure the success of the team. With each weekly team meeting we would take time to reflect on how we resolved issues and while looking at the strengths and weaknesses of our collaborative efforts. 
\end{itemize}


\section{Outline of each chapter}

\subsection{Context}
\subsection{Methodology}
\subsection{Technology Review}
\subsection{System Design}
\subsection{System Evaluation}
\subsection{Conclusion}