\chapter{Introduction}
During our first three years in GMIT (Galway Mayo Institute of Technology) we were taught a broad range of topics from hardware to software. This was done to get us ready for whatever facet we chose. A major part of this project is to show what we have learned and put it into practice.

In late 2018, in the first semester of our fourth year, we were told we would be doing a group project over the two semesters. So in October 2018 Kevin Barry and Cian Gannon decided to form a group and started brainstorming ideas in order to start work early. Bringing those ideas to our project supervisors and getting their feedback on our idea we were able to form the idea for our final year project. With input from supervisors and using our own ideas, we were able to move forward with the technologies that we would use as our foundation. This was an extremely important decision for us as we wanted to pick relevant technologies and keep the scope within the scope of the level 8 course. 

In our first week of the first year, our student union representatives created a group page on Facebook. This was something which kept everyone in the course in contact with one another outside of college hours. Social media has rapidly expanded in the last ten years, where it's now quite hard to find someone who doesn't have some sort of social media account on one of the numerous platforms. Our world has become ever increasingly dependent on social media and the features it brings.

Looking at social media and how reliant our world has become on it, we wanted to understand it better and have a stronger grasp on the underlying technologies.

\section{Project Objectives} \label{objectives}
As mentioned above, our main goals for this project were to increase our understanding of how social media sites operate, the technologies used to make them and by changing how social media is designed by focusing at a specific target group.

This project is divided into two parts, the research-based dissertation, and the applied project. We will be discussing the project in terms of the research behind it and the technologies used to build it. For this reason, we split the objectives into dissertation or applied project based. We defined the objectives at a high level allowing us to break each objective into smaller tasks for development.

The objectives set out for the dissertation 
\paragraph{Dissertation}
\begin{itemize}
\item \textit{\textbf{Introduce the concept of the project}}. We will provide the reader with an introduction that describes the concept of the project, detailing its inspiration and goals.
\item \textit{\textbf{Provide an understanding of social media}}. Through extensive research, we will examine and outline the concept of social media. We will investigate a vast variety of topics ranging from how social media began, how it became what it is today and what we must do to stay relevant in the modern world.
\item \textit{\textbf{Provide an understanding of web technologies}}. To date, the vast number of technologies available for web development is expanding at a rapid rate. Exploring a range of different technologies we will discuss the pros and cons and give an insight into why we felt our final decision was most suitable for our project.
\item \textit{\textbf{Describe the development of the applied project}}. This dissertation aims to give the reader a comprehensive guide to the development process from the first steps of initial research to the end product. We will examine the approach of the team to the applied project, including methodologies and technologies used along with the design and evaluation of the system. To conclude we will evaluate the project and discuss any issues that occurred, how they were solved and what we would do differently in future development.
\end{itemize}
 
 \paragraph{Applied Project}
\begin{itemize}
\item \textit{\textbf{Produce a simple, easy to use web application}}. The project will implement numerous complex algorithms with a sophisticated API server and database at the back end. The final product must hide all the complexities from the user with a functional and appealing front end. Regardless of the user's technical abilities, the web application will be simple to use and navigate.
\item \textit{\textbf{Deliver a social platform for tech-savvy people that differs from the norm}}. Changing the way social media applications are generally developed, we will take a different approach and aim to deliver a solution by developing a platform with the end goal of delivering a social experience that targets a specific user base. The final application will provide an immerse online community where like-minded people can connect, share posts and interact with each other by commenting and giving feedback.
\item \textit{\textbf{Dive into new web technologies}}. For the purpose of expanding the team's knowledge portfolio and skill-set, the development will be conducted using new technologies that we previously have not explored. Maximizing the learning outcomes of the module by implementing modern practices to result in a streamlined up-to-date application.
\item \textit{\textbf{Complete the project collaborating as a team using an efficient and effective approach}}. The project will be developed as a collaborative effort. Using appropriate tools, methodologies and industry standards available to us, the project will be developed with the aim of simulating real industry experience. With the intention of maximizing efficiency and effectiveness, we will work closely swapping ideas, solving issues and reviewing performance to produce the optimal outcome.
\end{itemize}


\section{Metrics for success or failure} \label{metrics}
In order to create the application in a controlled manner and allow us to easily  manage progress in the early stages of this project it was crucial that we outlined the metrics for success or failure. Doing so, enabled us to keep on track with the core focus of what we were attempting to achieve. The metrics we defined related closely to the objectives outlined in Section \ref{objectives}. The simplified list of metrics for the success of both a dissertation and web application perspective is as follow:
\begin{itemize}
  \item \textit{A concise, yet comprehensive dissertation which can be understood by anyone regardless of the initial knowledge of the technologies implemented.} To measure this, as each section of the dissertation was roughly drafted up, we released sections to friends and fellow students to read. We would then process their feedback and make alterations were required.  
  \item \textit{A Simple, easy to use, functional web application}. To measure this we followed the same steps as above, releasing beta versions to friends of different technological abilities, allowing us to receive feedback both from a technical and non-technical perspective.
  \item \textit{A social platform that actually develops an online community}. To measure the effectiveness of the application from a social point of view. We kept a log of the user activities while monitoring the site. Thus enabling us to at certain time points be able to see users following more users and interacting with a larger reach.
  \item \textit{Teamwork Collaboration} We felt from a project management perspective and by researching the importance of a good team in industry, that we would measure the success of the team. With each weekly team meeting, we would take time to reflect on how we resolved issues, while looking at the strengths and weaknesses of our collaborative efforts. 
\end{itemize}


\section{Outline of each chapter}
This paper has been organized into different chapters. Each chapter contains different details regarding various aspects of the project. The following sub-sections will briefly outline each chapter.

\subsection{Context}
In \textit{Chapter \ref{context}} we will investigate how social media has expanded to play a significant role in society today. We will research a wide variety of topics relating to the topic. Starting with \textit{Early Electronic Communication} we investigate how the internet was formed, giving us the base to develop online social platforms. Examining social media from its earliest days we discuss the \textit{Rise of Social Media}. Finally, this chapter concludes by discussing \textit{Modern Social Media} and how to \textit{Stay Relevant in our Ever-changing World}.

\subsection{Methodology}
\textit{Chapter \ref{methodology}} will explore the approaches followed to plan, organize, manage and develop the project. We will discuss the methodologies that were adapted and combined to complete the research and development of the project along with why they were implemented. This section aims to give insight to the reader how the project transformed from research to final software while collaborating as a team.

\subsection{Technology Review}
In \textit{Chapter \ref{techreview}} we will cover the technical side of our project, looking back on the technologies that made up the final revision of the project. We will explain the different technologies we added and how they were implemented through the project. We will look over the web stack we used and the technologies we added in order to create a more robust and useful social media site. We will go over why we used the given technologies and the benefits we saw in them over others.

\subsection{System Design}
In this chapter, we will discuss the architecture and design of the \textbf{TechBook} system. Presenting code snippets and visual diagrams to help portray a basic understanding of the application design. The contents of this chapter will begin with a brief overview of the flow of the architecture followed by a more in-depth portrayal separated into the Data Tier, Logic Tier and Presentation Tier.

\subsection{System Evaluation}
This chapter will evaluate the software developed in the project. We will evaluate the system in the areas of robustness, testing, and scalability. We will measure the results of the system against the objectives specified in the introduction. It will also highlight the limitations of the software and analyze where there are opportunities to improve the approach and technologies used,

\subsection{Conclusion}
To conclude we will briefly review the overall rationale and goals of the project. Highlighting our findings from  Chapter \ref{eval} \textit{System Evaluation}. Giving a final analysis, we review our discoveries gained from research and the new skills acquired as a result. Finally, we will finish on a positive note with a brief discussion of the team's experience of the project.

\section{Requirement Specification}
Below is a list of the requirements of the project:

\paragraph{User Requirements} \label{userreq}
\begin{itemize}
    \item Must be able to log in and log out.
    \item Log in will be persistent throughout a session. 
    \item Must be able to register an account.
    \item Must be able to edit account details.
    \item Must be able to upload and edit profile follow.
    \item Allow user to follow other users profiles.
    \item Allow user to be followed by other profiles
    \item User can upload a post.
    \item User can comment on posts.
    \item Users credentials must be stored safely and encrypted on the database.
\end{itemize}