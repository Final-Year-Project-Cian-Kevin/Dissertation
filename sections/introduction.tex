\chapter{Introduction}
 pages / 3-4000 words
What is it about? Is it at the right level (8)? Is the scope correct? Do not assume that the reader knows anything about the domain.

Why should a reader care or be interested?
Set out the objectives of the project clearly
You will have to address each of these in the evaluation / conclusion.
The metrics by which success or failure is measured.
Briefly list each chapter / section and provide a brief description of what each section contains.
List the resource URL (GitHub address) for the project and provide a brief list and description of the main elements at the URL.

After reading the introduction, a reader should be 100pc certain of what the project is all about and why it is relevant.
The first section should do three things: introduce the subject matter: explaining the motivation for the project, and saying something about the background area; describe the contribution made by the dissertation: the results of the project, the impact or value of the work; provide an outline of the dissertation, explaining how this contribution is realised in the subsequent sections.

This section may be only a page or two in length, but it is the most important part of the dissertation. It will shape the expectations of the reader, and provide them with a guide to the rest of the work.
Notes
\section{got from online}
The introduction has several purposes. Clearly one is to set the scene for the project by giving a little relevant background information - try to grab the reader's interest early. Another is to clearly elucidate the aims and objectives of the project and the constraints that might affect the way in which the project is carried out. If the project involves the solution of a specific problem or the production of a specific system this should be clearly specified in an informal way. Finally, the introduction should summarise the remaining chapters of the dissertation, in effect giving the reader an overview of what is to come.

The type of project will dictate the content and structure of the following chapters and you should discuss this with your supervisor. For example, for a theoretical project it is likely that several chapters will be devoted to constructing the theoretical foundations for the project and will consist of your own interpretation and synthesis of existing work with suitable examples discussed throughout. A sequence of chapters that cover theoretical framework, conditions and assumptions and theory application and comparisons may be appropriate. For an experimental project, the experimental goals, design, execution and evaluation might be covered. What now follows is a typical structure for a 'design and build' project. 

At the end of chapter 1, you should include a brief discussion of your view of the relationship between your project, and your degree programme. In your discussion, you should mention any advantages or challenges created by this relationship. 
 \section{Project Objectives}
 \section{Outline of each chapter}

