
\chapter{Conclusion}
\section{Overview}
\paragraph{Concluding} this dissertation, we will consider our aims and achievements of both the theoretical and applied components of our project while also complying  and full filling with user requirements. 

Our main goal of the project was to create a unique social media platform that provides and online community for tech savvy users. We have succeeded in doing this, by developing an easy to use, responsive web application that utilizes modern full stack web development tools to create a 3 tier architecture. As presented throughout this paper our final product is a fully functional platform that allow users to register, log in, create posts, follow users and so much more.

In \textit{Section \ref{objectives}} we proposed the following objectives:
\begin{itemize}
    \item Introduce the concept of the project.
    \item Provide an understanding of social media.
    \item Provide an understanding of web technologies.
    \item Describe the development of the applied project.
    \item Produce a simple, easy to use web application.
    \item Deliver a social platform for tech savvy people that differs from the norm.
    \item Dive into new web technologies.
    \item Complete the project collaborating as a team using an efficient and effective approach.  
\end{itemize}

The evidence from our findings discussed in \textit{Section \ref{eval} System evaluation} clearly demonstrates that the dissertation has achieved all its original objectives. We can conclude that we have provided the reader with an extensive yet clear insight into the world of social media. Our thorough investigation into all aspects of social media completed in\textit{Context \ref{context}} demonstrate the major importance of social media in modern society. The strength of this importance contributes to the overall rationale and decision choice of developing a social platform. 

Concluding the theoretical aspects of the dissertation, we dug deep into the research of web technologies and documenting the full development of the applied project. We have shown how the combination of modern technologies, advanced methodologies and best practices can be combined to the produce a fully functional modern web application.

Looking back at the applied objectives we are more than happy, achieving far more than we could have expected. \textbf{TechBook} offers a unique online community that can be navigated by all user regardless of their technological abilities. Harnessing the power of the MEAN stacks 3-tier architecture, we have produced a scalable, sophisticated API that handles the applications data load, while portraying a simplistic front end to the user hiding all of the complexities. 

From a collaborative perspective the project was completed with an efficient and effective approach. The result of our efforts, as a team, prove the massive significance of the importance of not only being technically skilled, but being capable to work alongside others in order to achieve the optimum result. We functioned very well as a team, keeping communications professional while maintaining a friendly relationship. Solving all issues and differences of opinions as professionals with the projects end goals in mind. 

In terms  of the requirements stated in Section \textit{\ref{userreq} User Requirements} the end product complies with all the initial user requirements specified. Allowing a vast variety of functionality to the user. The project hits all check boxes for what a social media platform requires to become a community by allowing users to register, make friends, follow other users and posts, share and give feedback on other users content.

\section{Learning Outcomes}
\paragraph{Throughout} the entire scope of the project from start to finish, we have gathered a substantial amount of knowledge on a vast variety of subjects and technologies. From initial research and design to implementing a MEAN stack 3-tier application with a scale-able API.  Pushing our abilities and immersing ourselves into new technologies to achieve the learning outcomes of the module. The skills obtained and experiences earned will carry into our careers, being applied to industry practice. Our research gave us the awareness and ability to explore present state of the art computing areas and evaluate the literature base. We seen first hand the importance of critically evaluating the work and research required to complete a project.

\paragraph{From} a team perspective the skills learned are priceless. It is clear that everyone works differently and no two developers are the same. Utilising the strengths and weaknesses of different members of a team from a management and personal level is something so important. This is something that is not easily thought but a crucial skill we acquired through experience. It became evident that as the project grew in scope, the better the communication the better the result. The module has educated us on the importance of communication is vital in software development to ensure work can be completed quicker and deadlines can be met.

\section{Final Thoughts} 
\paragraph{As} we come to the end of the project and our final weeks in \textit{GMIT}, we hand up this project with a smile on our faces, satisfied that we have achieved our goals, not only with this project but also in our academic career over the past four years. This project was challenging from the start but reaching the end goal has been such a rewarding experience.

\section{A Word From The Developers}

\begin{quote}
\textit{"Where to start..? This whole project has been an amazing experience from every perspective. Working along side Cian has been an absolute pleasure. To think that 5 years ago we met in GTI as strangers doing a level 5 I.T. course, we could have never imagined working together. From our early days of learning about computers to recently both signing contracts with massive organizations in recent weeks. Shout out to Dr.John Healy and Martin Kenirons for all their inspiration and motivation with the project. To rap this up, to whoever is reading this, I hope you enjoyed our project as much as we did."}\par\raggedleft--- \textup{Kevin Barry}
\end{quote}

\begin{quote}
\textit{"A great way to end my years as a student. I began studying programming 5 years ago in GTI, in the same year as Kevin. I still remember learning the basics (Strings, arrays, for loops) and thinking 'How would I ever be able to build an application?'. This project is a milestone I've reached in my development as a software developer, that I didn't think I'd reach when I started. This project is a great note to end on in my years of studying software development. With help from our supervisors, Dr.John Healy and Martin Kenirons who were extremely supportive during the development process."}\par\raggedleft--- \textup{Cian Gannon}
\end{quote}
