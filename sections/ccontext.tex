\chapter{Context}
In most cases, there will be a second section that describes the context of the work: the application area, the problem domain, or the theory that is to be extended or analysed. In other cases, if this description is only a couple of pages long, it could be made part of the Introduction.

This description should be limited to the context, not the work itself. If the project involves the creation of a model for a software product, then this section should describe the product, but not the model. If the project is theory-based, this description should describe the existing theory, but not the proposed extensions.

As with the Introduction, this section will shape the reader's expectations. It should provide enough information about the context to allow the reader to assess the contribution made. It should mention only those aspects or features that are relevant to the subsequent sections, emphasising those that are particularly relevant.

\begin{itemize}
\item Provide a context for your project.
\item Set out the objectives of the project
\item Briefly list each chapter / section and provide a 1-2 line description of what each section contains.
\item List the resource URL (GitHub address) for the project and provide a brief list of the main elements at the URL.
\end{itemize}
