\chapter{Technology Review}
This chapter will cover the technical side of our project, by looking back on the technologies that made up the final revision of the project. We will explain the different technologies we added and how they were implemented through the project. We will look over the web stack we used and the technologies we added in order to create a more robust and useful social media site. We will go over why we used the given technologies and the benefits we saw in them over others.

% ========================== Frame Work  ========================== 
\section{MEAN STACK/Framework}
The mean stack is intended to provide a simple and fun starting point for cloud native full-stack JavaScript applications. MEAN is a set of Open Source components that together, provide an end-to-end framework for building dynamic web applications; starting from the top (code running in the browser) to the bottom (database). The stack is made up of:

\begin{itemize}
\item MongoDB
\item ExpressJS
\item Angular
\item Node.js
\end{itemize}

Early in development when we were brainstorming, we decided to do a full stack development. The next step after this decision was to figure out how to carry out this task. We looked into full development stacks such as the MERN stack (MongoDB, ExpressJS, React, Node.js), MEAN stack (MongoDB, ExpressJS, Angular, Node.js) or doing a Java-based development stack (Spring boot, Angular, MongoDB). After lots of discussion and input from supervisors we decided to not go with the Java-based development stack. We then took a deep dive into the difference between the MEAN/MERN stack to decide what route we wanted to take and start development as soon as possible. After looking into React JSX vs Angular HTML templating and experimenting with the two stacks we finally decided that the MEAN stack was more inline in what we wanted to learn.

\subsection{MongoDB}
Document database – used by your back-end application to store its data as JSON (JavaScript Object Notation) documents

\subsection{Express}
sometimes referred to as Express.js): Back-end web application framework running on top of Node.js

\subsection{Angular}
(formerly Angular.js): Front-end web app framework; runs your JavaScript code in the user’s browser, allowing your application UI to be dynamic

\subsection{Node.js}
JavaScript runtime environment – lets you implement your application back-end in JavaScrip
% ========================== DEPLOYMENT  ========================== 
\section{DEPLOYMENT}
The Technolog used to deploy . the diff options avaliable 
\subsection{AWS cloud 9}
why did we use aws?

% ========================== Data Gathering  ========================== 

\section{Data Gathering}
Where did we get site data?
\subsection{Reddit api}
why reddit ?

What was our aim in styling? research of other social platforms.

% ========================== Styling and UI  ========================== 
\section{Styling and UI}
What was our aim in styling? research of other social platforms.

\subsection{CSS}
why?
\subsection{Bootstrap}
why?
\subsection{Angular Material}
why?

% ========================== Other Technology's  ========================== 
\section{Other Technology's/dev environment}
Other technology used
\subsection{Visual Studio Code}
\subsection{Browsers}
chrome firefox
\subsection{Swagger}
\subsection{Latex}



